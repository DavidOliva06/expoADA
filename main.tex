\documentclass[10pt,aspectratio=169]{beamer}
%\usepackage{poly}
\usepackage{booktabs} 
\usepackage{listings}
\usepackage{algorithm}
\usepackage{algpseudocode}

% ================================================================================
% Metadata
% ================================================================================

\title{Iterated Local Search (ILS)}
\subtitle{Metaheurísticas iterativas para optimización combinatoria}
\author{Integrantes: }
\institute[ADA]{Curso ADA – Análisis y Diseño de Algoritmos}
\date{\today}

% ================================================================================
% Main Body
% ================================================================================

\begin{document} 
\maketitle
% ================================================================================
\section{Integrante 1: Introducción y fundamentos}
% ================================================================================
% ================================================================================
\section{Integrante 2: Estructura general del ILS}
% ================================================================================
%

\section{Integrante 3: Aplicación del ILS al FSSP}
% ================================================================================

\section{Aplicación del ILS al FSSP}

%------------------ Mapa de tu sección -------------------
\begin{frame}{Que veremos en esta seccion}
  \begin{itemize}
    \item ¿Qué es el \textbf{Flow Shop Scheduling Problem (FSSP)}?
    \item Cómo se adapta \textbf{ILS} para resolver FSSP
    \item \textbf{Solución inicial}: heurística NEH
    \item \textbf{Búsqueda local}: vecindarios (reinserción, intercambio)
    \item \textbf{Perturbación}: cambio fuerte para escapar de óptimos locales
    \item \textbf{Pseudocódigo} de ILS-FSSP y métricas
  \end{itemize}
\end{frame}

%------------------ ¿Qué es FSSP? ------------------------
\begin{frame}{¿Qué es el FSSP?}
  \begin{itemize}
    \item Tenemos $n$ \textbf{trabajos} (tareas/productos) que deben procesarse en $m$ \textbf{máquinas}.
    \item Todos los trabajos pasan por las máquinas \textbf{en el mismo orden}: $M_1 \rightarrow M_2 \rightarrow \dots \rightarrow M_m$.
    \item Objetivo típico: \textbf{minimizar el makespan} $C_{\max}$ (tiempo total para terminar todos los trabajos).
    \item \textbf{Difícil} porque el número de posibles secuencias es $n!$ (explota rápidamente).
  \end{itemize}

  % TODO: reemplaza por una imagen si quieres
  \begin{center}
    \fbox{\parbox{0.9\linewidth}{\centering \textbf{TODO: imagen}\\
    Diagrama de flujo de trabajos por máquinas: $M_1 \rightarrow M_2 \rightarrow M_3$}}
  \end{center}
\end{frame}

%------------------ Ejemplo Panadería --------------------
\begin{frame}{Ejemplo cotidiano: Panadería (3 máquinas)}
  \small
  \begin{center}
  \begin{tabular}{lccc}
    \toprule
    \textbf{Pan} & Amasar $(M_1)$ & Hornear $(M_2)$ & Empaquetar $(M_3)$ \\
    \midrule
    A & 5 & 7 & 2 \\
    B & 4 & 3 & 5 \\
    C & 6 & 5 & 3 \\
    \bottomrule
  \end{tabular}
  \end{center}
  \vspace{4mm}
  \begin{itemize}
    \item Queremos el \textbf{orden de panes} (A, B, C, ...) que minimiza el tiempo total $C_{\max}$.
    \item Evaluar todas las permutaciones es inviable cuando $n$ crece $\Rightarrow$ usamos \textbf{ILS}.
  \end{itemize}

  \begin{center}
    \fbox{\parbox{0.9\linewidth}{\centering \textbf{TODO: imagen}\\
    Foto/ícono de panadería o línea de producción}}
  \end{center}
\end{frame}

%------------------ ILS para FSSP (visión general) -------
\begin{frame}{Cómo ILS se aplica al FSSP (visión general)}
  \begin{enumerate}
    \item \textbf{Solución inicial}: construir una buena secuencia (NEH).
    \item \textbf{Búsqueda local}: mejorar la secuencia con vecindarios.
    \item \textbf{Perturbación}: realizar un \emph{cambio fuerte} para escapar de óptimos locales.
    \item \textbf{Aceptar/Rechazar}: decidir desde qué solución continuar.
    \item \textbf{Repetir} hasta cumplir el criterio de parada (tiempo/iteraciones).
  \end{enumerate}

  \begin{center}
    \fbox{\parbox{0.9\linewidth}{\centering \textbf{TODO: imagen}\\
    Infografía de ciclo: \emph{NEH} $\rightarrow$ BL $\rightarrow$ Perturbación $\rightarrow$ Aceptación $\rightarrow$ Repetir}}
  \end{center}
\end{frame}

%------------------ NEH como solución inicial ------------
\begin{frame}{Solución inicial con NEH}
  \begin{itemize}
    \item \textbf{NEH (Nawaz–Enscore–Ham)} es una heurística constructiva muy efectiva para FSSP.
    \item Idea resumida: ordenar trabajos por “peso” (suma de tiempos), e insertar iterativamente en la posición que \textbf{minimiza} $C_{\max}$ parcial.
    \item Ventaja: parte de una secuencia \textbf{ya buena}, acelerando ILS.
  \end{itemize}

  \begin{center}
    \fbox{\parbox{0.9\linewidth}{\centering \textbf{TODO: imagen}\\
    Diagrama simple: orden por suma de tiempos y construcción por inserciones óptimas}}
  \end{center}
\end{frame}

%------------------ Búsqueda local: vecindarios ----------
\begin{frame}{Búsqueda local: vecindarios en FSSP}
  \begin{itemize}
    \item \textbf{Reinserción}: quitar un trabajo de la posición $i$ y colocarlo en la posición $j$.
    \item \textbf{Intercambio} (swap): intercambiar las posiciones de dos trabajos.
  \end{itemize}

  \begin{center}
    \fbox{\parbox{0.9\linewidth}{\centering \textbf{TODO: imagen}\\
    Bocetos de reinserción vs intercambio en una secuencia [\;1\;2\;3\;4\;5\;]}}
  \end{center}
\end{frame}

%------------------ Perturbación -------------------------
\begin{frame}{Perturbación (cambio fuerte)}
  \begin{itemize}
    \item Cuando la BL ya no mejora (\textbf{óptimo local}), aplicamos una \textbf{perturbación}:
      \begin{itemize}
        \item Ejemplo típico: \textbf{intercambiar dos pares aleatorios de trabajos}.
        \item O invertir un bloque largo, o múltiples reinserciones encadenadas.
      \end{itemize}
    \item Objetivo: \textbf{escapar} del óptimo local para explorar otra región de soluciones.
    \item Luego se ejecuta \textbf{búsqueda local} sobre la solución perturbada.
  \end{itemize}

  \begin{center}
    \fbox{\parbox{0.9\linewidth}{\centering \textbf{TODO: imagen}\\
    Visual: secuencia antes/después de un cambio fuerte}}
  \end{center}
\end{frame}

%------------------ Pseudocódigo ILS-FSSP ----------------
\begin{frame}{Pseudocódigo resumido de ILS para FSSP}
\small
\begin{algorithm}[H]
\caption{ILS-FSSP (resumen)}
\begin{algorithmic}[1]
\State $s \gets$ NEH() \Comment{Solución inicial “buena”}
\State $s \gets$ BusquedaLocal($s$) \Comment{Reinserción / Intercambio}
\State $sb \gets s$ \Comment{Mejor global}
\While{\textit{tiempo $<$ límite} \textbf{ o } \textit{iter $<$ maxIters}}
  \State $s' \gets$ Perturbar($s$) \Comment{p.ej., 2 swaps aleatorios}
  \State $s' \gets$ BusquedaLocal($s'$)
  \If{$C_{\max}(s') < C_{\max}(sb)$}
     \State $sb \gets s'$
  \EndIf
  \State $s \gets$ CriterioAceptacion($s$, $s'$) \Comment{Better / RandomWalk / SA}
\EndWhile
\State \Return $sb$
\end{algorithmic}
\end{algorithm}
\end{frame}




% ================================================================================
\section{Integrante 4: Criterios de aceptación y variantes}
% =====================================================

% =======================CRITERIOS==============================
\begin{frame}{Criterios de aceptación en ILS}
\small
En \textbf{ILS}, el criterio de aceptación decide cuándo una nueva solución  
reemplaza a la actual, equilibrando la \textit{exploración} y la \textit{explotación} del espacio de búsqueda.


\bigskip
\textbf{Tipos principales de criterios:}
\begin{itemize}
    \item \textbf{Better} — Fomenta la \textit{intensificación}: solo acepta mejoras directas.  
    \item \textbf{Random Walk} — Promueve la \textit{diversificación}: acepta cualquier solución.  
    \item \textbf{Simulated Annealing} — Busca un \textit{equilibrio}: acepta soluciones peores con cierta probabilidad dependiente de la temperatura.
\end{itemize}

\medskip
\begin{block}{\centering }
--> Cada criterio ajusta el \textbf{balance} entre 
\textcolor{blue!70!black}{\textit{explorar}} nuevas regiones del espacio de búsqueda 
y \textcolor{orange!70!black}{\textit{explotar}} las mejores soluciones conocidas.
\end{block}
\end{frame}


% --------------------Pseudocódigo del ILS_B---------------------------------
% -----------------------------------------------------
\begin{frame}{Pseudocódigo del ILS\_B (criterio Better — intensificación)}

\begin{columns}[T]

% --------------------- COLUMNA IZQUIERDA: Código ---------------------
\begin{column}{0.45\textwidth}
\begin{algorithm}[H]
\caption{ILS\_B — Criterio de aceptación Better}
\begin{algorithmic}[1]
\State $s \gets$ NEH()
\State $s \gets$ BúsquedaLocal($s$)
\State $s_b \gets s$
\While{tiempo $<$ límite}
	\State $s \gets$ Perturbar($s_b$)
	\State $s \gets$ BúsquedaLocal($s$)
	\If{$C_{max}(s_b) > C_{max}(s)$}
		\State $s_b \gets s$
	\Else
		\State $s \gets s_b$
	\EndIf
\EndWhile
\State \Return $s_b$
\end{algorithmic}
\end{algorithm}
\end{column}

\begin{column}{0.55\textwidth}
\small
\textbf{Descripción general:}\\
El algoritmo ILS\_B maneja dos soluciones principales:  
\begin{itemize}
  \item $s$: la \textit{solución actual}.  
  \item $s_b$: la \textit{mejor solución encontrada hasta el momento}.  
\end{itemize}

\medskip
\textbf{Funcionamiento:}\\[-4pt]
\begin{itemize}
  \item En cada iteración, se genera una nueva solución perturbando $s_b$.  
  \item Se aplica una búsqueda local para mejorarla.  
  \item Si la nueva solución $s$ mejora el valor de $C_{max}$ respecto a $s_b$,  
        entonces $s_b$ se actualiza con $s$.  
  \item Si no mejora, se descarta $s$ y se vuelve a usar $s_b$.  
\end{itemize}

\medskip
   \begin{columns}[T,totalwidth=\textwidth]
% -------- COLUMNA 1: Criterio --------
   \begin{column}{0.48\textwidth}
    \textbf{Criterio “Better”:}\\[-4pt]
Solo se aceptan soluciones que sean mejores.  
La búsqueda se concentra cerca de la mejor solución conocida.  

   \end{column}

% -------- COLUMNA 2: Efecto --------
  \begin{column}{0.48\textwidth}
\textbf{Efecto:}\\[-4pt]
\begin{itemize}
  \item �� Alta intensificación.  
  \item �� Baja exploración.  
  \item  Riesgo de óptimos locales.  
   \end{itemize}
   \end{column}
   \end{columns}
  \end{column}
 \end{columns}

\end{frame}

% ----------------Pseudocódigo del ILS_RW -------------------------------------

\begin{frame}{Pseudocódigo del ILS\_RW con criterio de aceptación \textit{Random Walk}}
\begin{columns}[T]
\begin{column}{0.55\textwidth}

\begin{algorithm}[H]
\caption{ILS\_RW — Criterio de aceptación Random Walk}
\begin{algorithmic}[1]
\State $s \gets$ NEH()
\State $s \gets$ BúsquedaLocal($s$)
\State $s_b \gets s$
\While{tiempo $<$ límite}
    \State $s \gets$ Perturbar($s$)
    \State $s \gets$ BúsquedaLocal($s$)
    \If{$C_{max}(s_b) > C_{max}(s)$}
        \State $s_b \gets s$
    \EndIf
    \Comment{No se restaura $s = s_b$}
\EndWhile
\State \Return $s_b$
\end{algorithmic}
\end{algorithm}

\end{column}
\begin{column}{0.43\textwidth}
\smallskip
\textbf{Descripción general:}\\
Este enfoque se asemeja al ILS\_B,  
utiliza una solución actual $s$ y la mejor $s_b$,  
pero cambia el criterio de aceptación.  

\medskip
\textbf{Criterio “Random Walk”:}\\
Se aceptan todas las soluciones, sean mejores o peores.  
Elimina la restauración de $s = s_b$,  
lo que permite continuar la búsqueda  
desde la última solución encontrada.  

\medskip
\textbf{Efecto:}\\
Promueve una búsqueda más \textit{aleatoria y amplia},  
incrementando la \textbf{diversificación} del espacio de búsqueda ��.  
Sin embargo, puede perder calidad temporalmente  
al aceptar soluciones peores.
\end{column}
\end{columns}
\end{frame}


% ------------------------Pseudocódigo del ILS_SA -----------------------------
\begin{frame}{Pseudocódigo del ILS\_SA (criterio Simulated Annealing — I)}

\begin{columns}[T]

\begin{column}{0.55\textwidth}
\small
\textbf{Descripción general:}\\
ILS\_SA maneja tres soluciones principales:
\begin{itemize}
  \item $s$: \textit{solución actual.}\\[-6pt]
  \item $s'$: \textit{nueva solución generada.}\\[-6pt]
  \item $s_b$: \textit{mejor solución encontrada.}\\[-6pt]
\end{itemize}

\medskip
\textbf{Funcionamiento:}\\[-4pt]
\begin{itemize}
  \item Parte de una solución inicial generada por NEH.\\[-6pt]  
  \item Aplica búsqueda local para mejorarla.\\[-6pt]
  \item En cada iteración:
  \begin{itemize}
    \item Copia la solución actual $s$ en $s'$.  
    \item Perturba $s'$ y luego la mejora con búsqueda local.  
    \item Compara el valor de $C_{max}$ entre $s$ y $s'$.  
  \end{itemize}
\end{itemize}

\medskip
\textbf{Idea principal:}\\[-2pt]
El criterio \textbf{Simulated Annealing (SA)} decide si reemplazar la solución actual $s$ por la nueva $s'$  
según una probabilidad que depende de la \textit{temperatura} $T$.

\end{column}

\begin{column}{0.55\textwidth}
\begin{algorithm}[H]
\caption{ILS\_SA — Simulated Annealing}
\begin{algorithmic}[1]
\State $s \gets$ NEH()
\State $s \gets$ BúsquedaLocal($s$)
\State $s_b \gets s$
\While{tiempo $<$ límite}
	\State $s' \gets$ Copiar($s$)
	\State $s' \gets$ Perturbar($s', d$)
	\State $s' \gets$ BúsquedaLocal($s'$)
	\If{$C_{max}(s) > C_{max}(s')$}
		\State $s \gets s'$
		\If{$C_{max}(s_b) > C_{max}(s)$}
			\State $s_b \gets s$ 
            \EndIf
	\ElsIf{$random \leq e^{-(C_{max}(s') - C_{max}(s))/T}$}
		\State $s \gets s'$ 
\EndWhile
\State \Return $s_b$
\end{algorithmic}
\end{algorithm}
\end{column}
\end{columns}

\end{frame}

\begin{frame}{Pseudocódigo del ILS\_SA (criterio Simulated Annealing — II)}
\begin{columns}[T]



\begin{column}{0.55\textwidth}
\small
\textbf{Criterio de aceptación (SA):}\\[-4pt]
\begin{itemize}
  \item Si $s'$ es \textbf{mejor}, se reemplaza $s = s'$.\\[-4pt]
  \item Si $s'$ es \textbf{peor}, puede aceptarse con una probabilidad:
\end{itemize}

\[
P = e^{-\frac{C_{max}(s') - C_{max}(s)}{T}}
\]

donde $T$ es la \textbf{temperatura}, que controla el equilibrio entre  
\textit{exploración} e \textit{intensificación}.\\[4pt]

\textbf{Interpretación:}\\[-4pt]
\begin{itemize}
  \item Con $T$ alta → mayor probabilidad de aceptar soluciones peores.  
  \item Con $T$ baja → se aceptan solo mejoras claras.  
  \item Permite escapar de óptimos locales al inicio  
        y concentrarse luego en las mejores zonas.  
\end{itemize}

\end{column}

\begin{column}{0.45\textwidth}
\begin{algorithm}[H]
\caption{Criterio de aceptación SA}
\begin{algorithmic}[1]
\setcounter{ALG@line}{7}  % Comienza la numeración desde 8
\If{$C_{max}(s') < C_{max}(s)$}
    \State $s \gets s'$
    \If{$C_{max}(s_b) > C_{max}(s)$}
        \State $s_b \gets s$
    \EndIf
\ElsIf{$random \leq e^{-(C_{max}(s') - C_{max}(s))/T}$}
    \State $s \gets s'$
\EndIf
\State \Return $s_b$
\end{algorithmic}
\end{algorithm}
\end{column}

\end{columns}

\end{frame}



% -----------------------------------------------------
%====================COMPARACION=================================
\begin{frame}{Comparación de comportamientos entre los tres criterios}
\small
\begin{table}[H]
\centering
\renewcommand{\arraystretch}{1.3} % Espaciado entre filas
\begin{tabular}{p{3.3cm}ccc}
\toprule
\textbf{Característica} & \textbf{Better} & \textbf{Random Walk} & \textbf{Simulated Annealing} \\ 
\midrule
Acepta soluciones peores &  No &   Siempre &   A veces \\ 
Exploración global        & �� Baja & �� Alta & Media–Alta \\ 
Intensificación           & �� Alta & �� Baja &  Equilibrada \\ 
Riesgo de óptimo local    &  Alto &  Bajo &  Medio–Bajo \\ 
Control de probabilidad   & Simple & Nulo & Depende de $T$ \\ 
\bottomrule
\end{tabular}
\end{table}

\begin{block}{ }
El criterio de aceptación actúa como un \textbf{filtro adaptativo} dentro del ILS,  
determinando cómo se equilibra la \textit{exploración} y la \textit{intensificación}.  
Un balance adecuado mejora la calidad y diversidad de las soluciones obtenidas. ��
\end{block}
\end{frame}
%=====================================================



% -----------------VENTAJAS Y DESVENTAJAS------------------------------------
\begin{frame}{Ventajas y desventajas de los criterios}
\small
\begin{table}[H]
\centering
\begin{tabular}{p{2.8cm}p{4cm}p{4cm}}
\toprule
\textbf{Criterio} & \textbf{Ventajas} & \textbf{Desventajas} \\ \midrule
\textbf{Better} &
\begin{itemize}\item Simplicidad y rapidez.\item Asegura mejora continua.\end{itemize} &
\begin{itemize}\item Se estanca en óptimos locales.\item Poca exploración global.\end{itemize} \\ \midrule
\textbf{Random Walk} &
\begin{itemize}\item Explora todo el espacio de búsqueda.\item Escapa fácilmente de óptimos locales.\end{itemize} &
\begin{itemize}\item No garantiza mejora.\item Puede perder calidad de solución.\end{itemize} \\ \midrule
\textbf{Simulated Annealing} &
\begin{itemize}\item Balancea exploración e intensificación.\item Evita el estancamiento prematuro.\end{itemize} &
\begin{itemize}\item Requiere calibrar temperatura $T$.\item Mayor costo computacional.\end{itemize} \\ \bottomrule
\end{tabular}
\end{table}



\end{frame}





% ================================================================================
\section{Integrante 5: Análisis, conclusiones y referencias}
% ================================================================================

\begin{frame}{Ventajas generales del ILS}

\begin{itemize}
	\item Simplicidad estructural y gran eficiencia.
	\item Requiere pocos parámetros, con resultados competitivos.
	\item Alterna entre intensificación y diversificación.
	\item Adaptable a distintos contextos industriales y de investigación.
\end{itemize}
\pause
\begin{block}{Ejemplo práctico}
En un taller de manufactura, el ILS puede reducir el \textit{makespan} entre 10–15\% frente a una búsqueda local pura (Stützle, 1998).
\end{block}
\end{frame}

\begin{frame}{Limitaciones y desafíos}
\begin{itemize}
	\item No garantiza el óptimo global.
	\item Su eficacia depende del diseño de perturbación y criterio de aceptación.
	\item Requiere calibración específica según el tipo de problema.
\end{itemize}
\end{frame}

\begin{frame}{Pseudocódigo del ILS aplicado al FSSP}

\begin{algorithm}[H]
\caption{Iterated Local Search (ILS) para Flow Shop Scheduling Problem}
\begin{algorithmic}[1]
\State $s_0 \gets$ NEH\_Inicial()
\State $s_{best} \gets$ BúsquedaLocal($s_0$)
\For{$iter = 1$ hasta $MaxIter$}
	\State $s' \gets$ Perturbar($s_{best}$)
	\State $s'' \gets$ BúsquedaLocal($s'$)
	\If{CriterioAceptación($s_{best}, s''$)}
		\State $s_{best} \gets s''$
	\EndIf
\EndFor
\State \Return $s_{best}$
\end{algorithmic}
\end{algorithm}
\end{frame}
\begin{frame}{Ejercicio: Instancia pequeña (4 trabajos, 3 máquinas)}
\small
\textbf{Datos (tiempos de procesamiento $p_{j,m}$):}
\[
\begin{array}{c|ccc}
\text{Trabajo} & M_1 & M_2 & M_3 \\ \hline
J1 & 2 & 3 & 2 \\
J2 & 4 & 1 & 3 \\
J3 & 3 & 2 & 1 \\
J4 & 2 & 4 & 2 \\
\end{array}
\]

\bigskip
\textbf{Objetivo:} aplicar NEH $\rightarrow$ Búsqueda Local (swap/reinserción) $\rightarrow$ Perturbación $\rightarrow$ Criterio de aceptación (ILS\_B) y mostrar cálculos del \textit{makespan}.
\end{frame}

% ------------------------------------------------------------------------
\begin{frame}{Paso 1 — Construcción NEH}
\small
\begin{itemize}
	\item Calcular sumas por trabajo: \\
	$sum(J1)=7,\; sum(J2)=8,\; sum(J3)=6,\; sum(J4)=8$. \\
	Orden descendente (rompiendo empates por índice): $[J2, J4, J1, J3]$.
	\item Insertar progresivamente el siguiente trabajo en la mejor posición (evaluando \textbf{makespan}).
\end{itemize}

\bigskip
\textbf{Insertos (resumen de evaluaciones):}
\begin{itemize}
	\item Inicial: $[J2]$ $\Rightarrow$ makespan $=8$.
	\item Insertar $J4$ $\Rightarrow$ probar $[J4,J2]$ (makespan $=11$) y $[J2,J4]$ (makespan $=12$) $\Rightarrow$ elegir $[J4,J2]$ (11).
	\item Insertar $J1$ $\Rightarrow$ mejores opciones: $[J4,J2,J1]$ (makespan $=13$) $\Rightarrow$ elegir $[J4,J2,J1]$.
	\item Insertar $J3$ $\Rightarrow$ probar posiciones $\Rightarrow$ mejor secuencia final: $[J4,J2,J1,J3]$ con \textbf{makespan = 14}.
\end{itemize}
\end{frame}

% ------------------------------------------------------------------------
\begin{frame}{Cálculo del makespan (ejemplo corto)}
\small
\textbf{Cómo se calcula (reglas):} para secuencia $S$, tabla de finalización $C[i,m]$:
\[
\begin{aligned}
C[1,1] &= p_{S_1,1}, \\
C[1,m] &= C[1,m-1] + p_{S_1,m},\\
C[i,1] &= C[i-1,1] + p_{S_i,1},\\
C[i,m] &= \max(C[i-1,m],C[i,m-1]) + p_{S_i,m}.
\end{aligned}
\]

\bigskip
\textbf{Cálculo parcial para la secuencia final $[J4,J2,J1,J3]$:} (valores resumidos)
\[
\text{Makespan} = C[4,3] = 14.
\]
\end{frame}

% ------------------------------------------------------------------------
\begin{frame}{Paso 2 — Búsqueda Local (vecindarios probados)}
\small
\textbf{Vecindarios usados:} intercambio (swap) y reinserción (take-and-insert).

\bigskip
\textbf{Evaluación de vecinos inmediatos de $[J4,J2,J1,J3]$:}
\begin{itemize}
	\item Swap posiciones 1–2 $\rightarrow$ $[J2,J4,J1,J3]$ $\Rightarrow$ makespan $=16$ (peor).
	\item Swap posiciones 2–3 $\rightarrow$ $[J4,J1,J2,J3]$ $\Rightarrow$ makespan $=15$ (peor).
	\item Reinserciones similares evaluadas (todas muestran makespan $\ge 14$).
\end{itemize}

\bigskip
\textbf{Conclusión:} la solución NEH $[J4,J2,J1,J3]$ con makespan $=14$ es localmente óptima respecto a los vecinos evaluados.
\end{frame}

% ------------------------------------------------------------------------
\begin{frame}{Paso 3 — Perturbación y aceptación (ILS\_B)}
\small
\textbf{Estrategia de perturbación:} intercambio aleatorio de $k$ pares (aquí $k=2$) o reinserción de 1 trabajo en posición aleatoria.

\bigskip
\textbf{Ejemplo de perturbación:} intercambio aleatorio entre posiciones 2 y 4:
\[
[ J4, J2, J1, J3 ] \xrightarrow{\text{perturb}} [ J4, J3, J1, J2 ]
\]
makespan de la secuencia perturbada $\Rightarrow$ $=16$ (peor).

\bigskip
\textbf{Criterio de aceptación (ILS\_B):} aceptar solo si $makespan(s'') < makespan(s_{best})$. \\
Como la perturbación empeora ($16 \not< 14$), no se acepta y se restaura $s_{best}$.
\end{frame}

% ------------------------------------------------------------------------
\begin{frame}{Paso 4 — Resumen del procedimiento y observaciones}
\small
\begin{itemize}
	\item NEH produjo $s_{NEH} = [J4,J2,J1,J3]$ con \textbf{makespan = 14}.
	\item Búsqueda Local exhaustiva alrededor de $s_{NEH}$ no encontró mejor vecino (local óptimo).
	\item Perturbaciones moderadas aumentaron el makespan en este caso; con criterio ILS\_B no se aceptan.
	\item Para escapar de este óptimo local se podría usar ILS\_SA (aceptación probabilística) o ILS\_RW para diversificar.
\end{itemize}
\end{frame}

% ------------------------------------------------------------------------
\begin{frame}[fragile]{Pseudocódigo: resolución paso a paso del ejercicio}
\small
\begin{algorithm}[H]
\caption{ResolverInstanciaEjemplo()}
\begin{algorithmic}[1]
\State \textbf{Entrada:} Matriz $p_{j,m}$ (4 trabajos × 3 máquinas), $MaxIter$, criterio = ILS\_B
\State // 1. Solución inicial con NEH
\State $s \gets$ NEH\_Inicial($p$) \Comment{$s \leftarrow [J4,J2,J1,J3]$}
\State $ms\_s \gets$ Makespan($s$) \Comment{$ms\_s \leftarrow 14$}
\State // 2. Búsqueda local sobre $s$
\State $s'\!, ms\_s' \gets$ BúsquedaLocal($s$) \Comment{evaluar vecinos swap/reinsert}
\If{$ms\_s' < ms\_s$}
    \State $s \gets s'$; $ms\_s \gets ms\_s'$
\Else
    \State mantener $s$
\EndIf
\State // 3. Iteraciones ILS
\For{$iter \leftarrow 1$ \textbf{to} $MaxIter$}
    \State $s\_p \gets$ Perturbar($s$) \Comment{ej. swap(2,4)}
    \State $s\_p \gets$ BúsquedaLocal($s\_p$)
    \If{CriterioAceptación($s, s\_p$)} \Comment{con ILS\_B: aceptar solo si mejora}
        \State $s \gets s\_p$
        \State $ms\_s \gets$ Makespan($s$)
    \EndIf
\EndFor
\State \Return $s, ms\_s$
\end{algorithmic}
\end{algorithm}
\end{frame}

\begin{frame}{Conclusiones generales}
\begin{itemize}[]
	\item El ILS es un método robusto, flexible y eficiente.
	\item Su simplicidad lo hace ideal para problemas reales.
	\item Balancea correctamente exploración y explotación.
\end{itemize}

\pause
\begin{block}{Reflexión final}
“El verdadero valor del ILS no está en reinventar la búsqueda local, sino en enseñarle a escapar de sí misma.” — \textit{Lourenço et al., 2003}
\end{block}
\end{frame}

\begin{frame}{Referencias}
\footnotesize
\begin{thebibliography}{99}
\bibitem{stutzle1998} Stützle, T. (1998). \textit{Iterated local search for the quadratic assignment problem.} AIDA-98-04, Darmstadt University.
\bibitem{lourenco2003} Lourenço, H. R., Martin, O. C., \& Stützle, T. (2003). \textit{Iterated Local Search.} Handbook of Metaheuristics. Springer.
\bibitem{ruiz2005} Ruiz, R., \& Maroto, C. (2005). \textit{Permutation flowshop heuristics.} European Journal of Operational Research, 165(2), 479–494.
\bibitem{blum2003} Blum, C., \& Roli, A. (2003). \textit{Metaheuristics in combinatorial optimization.} ACM Computing Surveys, 35(3), 268–308.
\bibitem{prins2004} Prins, C. (2004). \textit{Evolutionary algorithm for VRP.} Computers \& Operations Research, 31(12), 1985–2002.
\bibitem{fernandez2017} Fernandez-Viagas, V., Ruiz, R., & Framinan, J.M. (2017). \textit{A new vision of approximate methods for the permutation flowshop to minimise makespan: State-of-the-art and computational evaluation.} European Journal of Operational Research, 257(3), 707–721.
\bibitem{ruiz2005} Ruiz, R., & Maroto, C. (2005). \textit{A comprehensive review and evaluation of permutation flowshop heuristics.} European Journal of Operational Research, 165(2), 479–494.

\end{thebibliography}
\end{frame}

% ================================================================================
\backmatter
% ================================================================================

\end{document}
